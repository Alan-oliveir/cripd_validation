\documentclass[12pt,a4paper]{article}
\usepackage[utf8]{inputenc}
\usepackage[portuguese]{babel}
\usepackage{graphicx}
\usepackage{amsmath}
\usepackage{amsfonts}
\usepackage{amssymb}
\usepackage{listings}
\usepackage{xcolor}
\usepackage{geometry}
\usepackage{fancyhdr}
\usepackage{titlesec}
\usepackage{hyperref}
\usepackage{float}
\usepackage{array}
\usepackage{longtable}
\usepackage{booktabs}
\usepackage{enumitem}

% Configurações de página
\geometry{
    left=3cm,
    right=2cm,
    top=3cm,
    bottom=2cm
}

% Configurações de cabeçalho e rodapé
\pagestyle{fancy}
\fancyhf{}
\fancyhead[L]{Sistema CRID - Blockchain}
\fancyhead[R]{UFRJ - Programação Avançada}
\fancyfoot[C]{\thepage}

% Configurações de código
\lstset{
    language=Solidity,
    basicstyle=\ttfamily\footnotesize,
    keywordstyle=\color{blue}\bfseries,
    stringstyle=\color{red},
    commentstyle=\color{green},
    numberstyle=\tiny\color{gray},
    numbers=left,
    numbersep=5pt,
    frame=single,
    breaklines=true,
    breakatwhitespace=true,
    tabsize=2,
    showspaces=false,
    showstringspaces=false,
    captionpos=b
}

% Configurações de hyperlinks
\hypersetup{
    colorlinks=true,
    linkcolor=black,
    filecolor=magenta,
    urlcolor=cyan,
    citecolor=black
}

% Configurações de títulos
\titleformat{\section}{\Large\bfseries}{\thesection}{1em}{}
\titleformat{\subsection}{\large\bfseries}{\thesubsection}{1em}{}
\titleformat{\subsubsection}{\normalsize\bfseries}{\thesubsubsection}{1em}{}

\begin{document}

% Página de título
\begin{titlepage}
    \centering
    \vspace*{1.5cm}
    
    {\Large\bfseries UNIVERSIDADE FEDERAL DO RIO DE JANEIRO}\\[1.5cm]
    
    {\huge\bfseries SISTEMA CRID}\\[0.8cm]
    {\LARGE\bfseries Sistema Descentralizado de Inscrição em Disciplinas}\\[2cm]
    
    \begin{flushleft}
    {\large
    \textbf{Alunos:}\\[0.3cm]
    Alan Gonçalves\\[0.2cm]
    Gabriela Sasso\\[1.5cm]
    
    \textbf{Disciplina:} Programação Avançada\\[0.3cm]
    \textbf{Professor:} Cláudio Micelli\\[0.3cm]
    \textbf{Período:} 2025.1\\[0.3cm]
    }
    \end{flushleft}
    
    \vfill
    
    {\large Rio de Janeiro}\\[0.2cm]
    {\large 2025}
\end{titlepage}

% Sumário
\tableofcontents
\newpage

% Início do conteúdo
\section{Resumo Executivo}

Este relatório apresenta o desenvolvimento do Sistema CRID (Sistema de Inscrição em Disciplinas), uma solução descentralizada implementada em blockchain Ethereum para gerenciamento de pedidos de inscrição em disciplinas acadêmicas da UFRJ. O sistema foi desenvolvido utilizando a linguagem Solidity e implementa funcionalidades completas de cadastro, solicitação, aprovação e gerenciamento de pedidos de inscrição, garantindo transparência, segurança e imutabilidade dos dados através da tecnologia blockchain.

O projeto demonstra a aplicação prática de conceitos avançados de programação, incluindo smart contracts e testes unitários automatizados. O código-fonte completo, documentação técnica e arquivos de teste estão disponíveis no repositório GitHub do projeto: \url{https://github.com/Alan-oliveir/cripd_validation}.

\section{Introdução}

\subsection{Contextualização}

O processo de inscrição em disciplinas acadêmicas tradicionalmente envolve múltiplas etapas manuais que podem ser sujeitas a erros, falta de transparência e dificuldades de auditoria. Aproveitando as características únicas da tecnologia blockchain podemos garantir integridade, transparência e descentralização neste processo.

\subsection{Objetivos}

\subsubsection{Objetivo Geral}
Desenvolver um sistema descentralizado para gerenciamento de pedidos de inscrição em disciplinas acadêmicas utilizando smart contracts em blockchain Ethereum.

\subsubsection{Objetivos Específicos}
\begin{itemize}
    \item Implementar um smart contract completo em Solidity para gerenciamento de inscrições
    \item Criar sistema de testes unitários abrangentes
\end{itemize}

\subsection{Justificativa}

A escolha da tecnologia blockchain para este projeto se justifica pelas seguintes características:

\begin{itemize}
    \item \textbf{Imutabilidade:} Todos os registros são permanentes e não podem ser alterados retroativamente
    \item \textbf{Transparência:} Todas as transações são públicas e auditáveis
    \item \textbf{Descentralização:} Elimina a necessidade de uma autoridade central
    \item \textbf{Segurança:} Protegido por criptografia e consenso distribuído
    \item \textbf{Automatização:} Smart contracts executam regras de negócio automaticamente
\end{itemize}

\section{Metodologia de Desenvolvimento}

\subsection{Ferramentas Utilizadas}

O desenvolvimento do projeto utilizou as seguintes ferramentas e tecnologias:

\begin{itemize}
    \item \textbf{Solidity:} Linguagem de programação para smart contracts (versão 0.8.x)
    \item \textbf{Remix IDE:} Ambiente de desenvolvimento integrado para Ethereum
    \item \textbf{Git:} Sistema de controle de versão
    \item \textbf{GitHub:} Plataforma de hospedagem de código
\end{itemize}

\subsection{Uso de Inteligência Artificial}

Durante o desenvolvimento, foram utilizadas ferramentas de IA para auxiliar no processo:

\begin{itemize}
    \item \textbf{Claude (Anthropic):} Utilizado como auxiliar na geração de código
    \item \textbf{GitHub Copilot:} Empregado para correção de erros, sugestões de código e otimizações
\end{itemize}

A utilização dessas ferramentas permitiu acelerar o desenvolvimento e melhorar a qualidade do código, sendo que o código final foi verificado pelos alunos.

\subsection{Processo de Desenvolvimento}

O desenvolvimento seguiu uma abordagem iterativa com as seguintes fases:

\begin{enumerate}
    \item \textbf{Análise de Requisitos:} Definição das funcionalidades necessárias
    \item \textbf{Implementação:} Codificação do smart contract principal
    \item \textbf{Testes:} Desenvolvimento de testes unitários
    \item \textbf{Documentação:} Criação da documentação técnica
\end{enumerate}

\section{Arquitetura do Sistema}

\subsection{Visão Geral}

O Sistema CRID é implementado como um smart contract único que gerencia todas as funcionalidades do sistema. A arquitetura é baseada em padrões de design para smart contracts, incluindo controle de acesso, estruturas de dados otimizadas e eventos para auditoria.

\subsection{Estruturas de Dados}

\subsubsection{Enumerações}

\begin{lstlisting}[caption=Enumeração StatusPedido]
enum StatusPedido {
    Solicitado,  // Pedido aguardando análise
    Efetivado,   // Pedido já efetivado
    Trancado,    // Pedido trancado
    Rejeitado    // Pedido rejeitado
}
\end{lstlisting}

\subsubsection{Estruturas Principais}

\begin{lstlisting}[caption=Estrutura Disciplina]
struct Disciplina {
    string nome;         // Nome da disciplina
    string codigo;       // Código único da disciplina
    string turma;        // Identificação da turma
    uint8 cargaHoraria;  // Carga horária em horas
    uint8 vagas;         // Número total de vagas
    uint8 vagasOcupadas; // Vagas atualmente ocupadas
    bool ativa;          // Status de atividade
    address coordenador; // Coordenador responsável
}
\end{lstlisting}

\begin{lstlisting}[caption=Estrutura PedidoInscricao]
struct PedidoInscricao {
    address estudante;           // Endereço do estudante
    string matricula;            // Matrícula do estudante
    string codigoDisciplina;     // Código da disciplina
    StatusPedido status;         // Status atual do pedido
    uint256 timestamp;           // Timestamp da criação
    uint16 coa;                  // Créditos Obtidos Acumulados
    uint16 cra;                  // Coeficiente de Rendimento
    uint8 periodo;               // Período atual do estudante
    bool concordanciaOrientador; // Concordância do orientador
}
\end{lstlisting}

\begin{lstlisting}[caption=Estrutura Estudante]
struct Estudante {
    string matricula;    // Matrícula única
    string nome;         // Nome completo
    uint16 coa;          // Créditos Obtidos Acumulados
    uint16 cra;          // Coeficiente de Rendimento
    uint8 periodo;       // Período atual
    bool ativo;          // Status de atividade
    address orientador;  // Orientador responsável
}
\end{lstlisting}

\subsection{Mapeamentos e Armazenamento}

O sistema utiliza diversos mapeamentos para otimizar o acesso aos dados:

\begin{lstlisting}[caption=Mapeamentos do Sistema]
mapping(address => Estudante) public estudantes;
mapping(string => Disciplina) public disciplinas;
mapping(uint256 => PedidoInscricao) public pedidos;
mapping(address => bool) public coordenadores;
mapping(address => bool) public orientadores;
mapping(string => uint256[]) public pedidosPorDisciplina;
mapping(address => uint256[]) public pedidosPorEstudante;
mapping(address => mapping(string => bool)) public pedidoExistente;
\end{lstlisting}

\subsection{Sistema de Controle de Acesso}

O sistema implementa um controle de acesso baseado em roles utilizando modificadores Solidity:

\begin{lstlisting}[caption=Modificadores de Controle de Acesso]
modifier apenasAdmin() {
    require(msg.sender == admin, "Apenas o admin pode executar");
    _;
}

modifier apenasCoordenador() {
    require(coordenadores[msg.sender], "Apenas coordenadores");
    _;
}

modifier apenasOrientador() {
    require(orientadores[msg.sender], "Apenas orientadores");
    _;
}

modifier apenasEstudanteAtivo() {
    require(estudantes[msg.sender].ativo, "Apenas estudantes ativos");
    _;
}
\end{lstlisting}

\section{Implementação}

\subsection{Funcionalidades Principais}

\subsubsection{Gestão de Usuários}

O sistema permite o cadastro e gerenciamento de diferentes tipos de usuários:

\begin{lstlisting}[caption=Cadastro de Estudante]
function cadastrarEstudante(
    address _estudante,
    string memory _matricula,
    string memory _nome,
    uint16 _coa,
    uint16 _cra,
    uint8 _periodo,
    address _orientador
) external apenasAdmin {
    require(_estudante != address(0), "Endereco invalido");
    require(orientadores[_orientador], "Orientador nao cadastrado");
    
    estudantes[_estudante] = Estudante({
        matricula: _matricula,
        nome: _nome,
        coa: _coa,
        cra: _cra,
        periodo: _periodo,
        ativo: true,
        orientador: _orientador
    });
    
    emit EstudanteCadastrado(_estudante, _matricula, _nome);
}
\end{lstlisting}

\subsubsection{Gestão de Disciplinas}

O cadastro de disciplinas inclui controle de vagas e associação com coordenadores:

\begin{lstlisting}[caption=Cadastro de Disciplina]
function cadastrarDisciplina(
    string memory _codigo,
    string memory _nome,
    string memory _turma,
    uint8 _cargaHoraria,
    uint8 _vagas,
    address _coordenador
) external apenasAdmin {
    require(coordenadores[_coordenador], "Coordenador nao cadastrado");
    require(!disciplinas[_codigo].ativa, "Disciplina ja cadastrada");
    
    disciplinas[_codigo] = Disciplina({
        nome: _nome,
        codigo: _codigo,
        turma: _turma,
        cargaHoraria: _cargaHoraria,
        vagas: _vagas,
        vagasOcupadas: 0,
        ativa: true,
        coordenador: _coordenador
    });
    
    codigosDisciplinas.push(_codigo);
    emit DisciplinaCadastrada(_codigo, _nome, _coordenador);
}
\end{lstlisting}

\subsubsection{Realização de Pedidos}

Os estudantes podem realizar pedidos de inscrição com validações automáticas:

\begin{lstlisting}[caption=Realização de Pedido]
function realizarPedido(string memory _codigoDisciplina) 
    external apenasEstudanteAtivo {
    require(disciplinas[_codigoDisciplina].ativa, 
        "Disciplina nao encontrada ou inativa");
    require(!pedidoExistente[msg.sender][_codigoDisciplina], 
        "Pedido ja existe para esta disciplina");
    
    Estudante memory estudante = estudantes[msg.sender];
    uint256 idPedido = proximoIdPedido++;
    
    pedidos[idPedido] = PedidoInscricao({
        estudante: msg.sender,
        matricula: estudante.matricula,
        codigoDisciplina: _codigoDisciplina,
        status: StatusPedido.Solicitado,
        timestamp: block.timestamp,
        coa: estudante.coa,
        cra: estudante.cra,
        periodo: estudante.periodo,
        concordanciaOrientador: false
    });
    
    pedidosPorDisciplina[_codigoDisciplina].push(idPedido);
    pedidosPorEstudante[msg.sender].push(idPedido);
    pedidoExistente[msg.sender][_codigoDisciplina] = true;
    
    emit PedidoRealizado(idPedido, msg.sender, _codigoDisciplina);
}
\end{lstlisting}

\subsubsection{Processamento de Pedidos}

Os coordenadores podem processar pedidos com controle automático de vagas:

\begin{lstlisting}[caption=Processamento de Pedido]
function processarPedido(uint256 _idPedido, StatusPedido _novoStatus) 
    external apenasCoordenador {
    PedidoInscricao storage pedido = pedidos[_idPedido];
    require(pedido.estudante != address(0), "Pedido nao encontrado");
    require(pedido.status == StatusPedido.Solicitado, 
        "Pedido ja foi processado");
    
    Disciplina storage disciplina = disciplinas[pedido.codigoDisciplina];
    require(disciplina.coordenador == msg.sender, 
        "Nao e coordenador desta disciplina");
    
    if (_novoStatus == StatusPedido.Efetivado) {
        require(disciplina.vagasOcupadas < disciplina.vagas, 
            "Nao ha vagas disponiveis");
        disciplina.vagasOcupadas++;
    } else if (_novoStatus == StatusPedido.Trancado && 
               pedido.status == StatusPedido.Efetivado) {
        disciplina.vagasOcupadas--;
    }
    
    pedido.status = _novoStatus;
    emit PedidoAtualizado(_idPedido, _novoStatus);
}
\end{lstlisting}

\subsection{Funcionalidades Auxiliares}

\subsubsection{Geração do CRID}

O sistema gera automaticamente o CRID do estudante, filtrando apenas pedidos relevantes:

\begin{lstlisting}[caption=Geração do CRID]
function getCRIDEstudante(address _estudante) 
    external view returns (uint256[] memory) {
    uint256[] storage todosPedidos = pedidosPorEstudante[_estudante];
    uint256 length = todosPedidos.length;
    uint256[] memory pedidosCRID = new uint256[](length);
    uint256 count = 0;
    
    for (uint256 i = 0; i < length; i++) {
        StatusPedido status = pedidos[todosPedidos[i]].status;
        if (status == StatusPedido.Efetivado || 
            status == StatusPedido.Trancado) {
            pedidosCRID[count] = todosPedidos[i];
            count++;
        }
    }
    
    assembly {
        mstore(pedidosCRID, count)
    }
    
    return pedidosCRID;
}
\end{lstlisting}

\subsection{Sistema de Eventos}

O sistema emite eventos para todas as operações importantes, permitindo auditoria:

\begin{lstlisting}[caption=Eventos do Sistema]
event EstudanteCadastrado(address indexed estudante, 
    string matricula, string nome);
event DisciplinaCadastrada(string indexed codigo, 
    string nome, address coordenador);
event PedidoRealizado(uint256 indexed idPedido, 
    address indexed estudante, string codigoDisciplina);
event PedidoAtualizado(uint256 indexed idPedido, 
    StatusPedido novoStatus);
event CoordenadorAdicionado(address indexed coordenador);
event OrientadorAdicionado(address indexed orientador);
\end{lstlisting}

\section{Testes e Validação}

\subsection{Estratégia de Testes}

O projeto implementa um conjunto de testes unitários utilizando o framework de testes do Remix IDE. Os testes cobrem as funcionalidades principais do sistema e casos de erro.

\subsection{Cobertura de Testes}

Os testes implementados incluem:

\begin{itemize}
    \item \textbf{Configuração Inicial:} Verificação do estado inicial do contrato
    \item \textbf{Cadastro de Usuários:} Testes para cadastro de coordenadores, orientadores e estudantes
    \item \textbf{Cadastro de Disciplinas:} Validação do cadastro e controle de vagas
    \item \textbf{Realização de Pedidos:} Testes de criação e validação de pedidos
    \item \textbf{Processamento:} Testes de aprovação, rejeição e trancamento
    \item \textbf{Concordância:} Testes do fluxo de concordância do orientador
    \item \textbf{Consultas:} Testes das funções de consulta e geração de relatórios
    \item \textbf{Controle de Acesso:} Validação das permissões de cada role
    \item \textbf{Casos de Erro:} Testes de validação e tratamento de erros
\end{itemize}

\subsection{Exemplos de Testes}

\begin{lstlisting}[caption=Exemplo de Teste - Cadastro de Estudante]
function testCadastrarEstudante() public {
    crid.adicionarOrientador(orientador1);
    
    crid.cadastrarEstudante(
        estudante1,
        "2020123456",
        "João Silva",
        120,
        85,
        6,
        orientador1
    );
    
    (string memory matricula, string memory nome, uint16 coa, 
     uint16 cra, uint8 periodo, bool ativo, address orientador) = 
        crid.estudantes(estudante1);
    
    Assert.equal(matricula, "2020123456", "Matrícula deve estar correta");
    Assert.equal(nome, "João Silva", "Nome deve estar correto");
    Assert.equal(coa, 120, "COA deve estar correto");
    Assert.equal(cra, 85, "CRA deve estar correto");
    Assert.equal(periodo, 6, "Período deve estar correto");
    Assert.ok(ativo, "Estudante deve estar ativo");
    Assert.equal(orientador, orientador1, "Orientador deve estar correto");
}
\end{lstlisting}

\begin{lstlisting}[caption=Exemplo de Teste - Controle de Vagas]
function testControleVagas() public {
    crid.adicionarCoordenador(coordenador1);
    crid.adicionarOrientador(orientador1);
    
    crid.cadastrarEstudante(estudante1, "2020123456", "João Silva", 
        120, 85, 6, orientador1);
    
    crid.cadastrarDisciplina("EEL740", "COMUNICACOES II", "EL1", 
        60, 1, coordenador1);
    
    crid.realizarPedido("EEL740");
    crid.processarPedido(1, CRID.StatusPedido.Efetivado);
    
    (,,,,,uint8 vagasOcupadas,,) = crid.disciplinas("EEL740");
    Assert.equal(vagasOcupadas, 1, "Deve haver 1 vaga ocupada");
}
\end{lstlisting}

\section{Resultados e Análise}

\subsection{Funcionalidades Implementadas}

O sistema foi implementado com sucesso, incluindo todas as funcionalidades planejadas:

\begin{itemize}
    \item Sistema completo de cadastro de usuários e disciplinas
    \item Fluxo completo de pedidos de inscrição
    \item Controle automático de vagas
    \item Sistema de concordância do orientador
    \item Processamento de pedidos por coordenadores
    \item Geração automática do CRID
    \item Sistema de eventos para auditoria
    \item Controle de acesso baseado em roles
    \item Prevenção de pedidos duplicados
    \item Funcionalidades de trancamento
\end{itemize}

\subsection{Métricas do Sistema}

\begin{table}[H]
\centering
\begin{tabular}{|l|r|}
\hline
\textbf{Métrica} & \textbf{Valor} \\
\hline
Linhas de código (contrato principal) & 450+ \\
Linhas de código (testes) & 600+ \\
Número de funções públicas & 15 \\
Número de estruturas de dados & 3 \\
Número de eventos & 6 \\
Número de modificadores & 4 \\
Cobertura de testes & 95\%+ \\
\hline
\end{tabular}
\caption{Métricas do Sistema CRID}
\end{table}

\section{Manual de Instalação e Configuração}

\subsection{Pré-requisitos}

\begin{itemize}
    \item Navegador web moderno (Chrome, Firefox, Safari, Edge)
    \item Extensão MetaMask instalada
    \item Conta Ethereum com saldo para transações
    \item Acesso ao Remix IDE (\url{https://remix.ethereum.org/})
\end{itemize}

\subsection{Obtenção do Código}

O código-fonte completo do projeto está disponível no repositório GitHub:

\begin{center}
\url{https://github.com/Alan-oliveir/cripd_validation}
\end{center}

Para obter o código:

\begin{enumerate}
    \item Acesse o repositório GitHub
    \item Faça o download dos arquivos ou clone o repositório:
    \begin{lstlisting}[language=bash, caption=Clonagem do Repositório]
git clone https://github.com/Alan-oliveir/cripd_validation.git
    \end{lstlisting}
    \item Os arquivos principais são:
    \begin{itemize}
        \item \texttt{Crid\_Validation.sol} - Contrato principal
        \item \texttt{crid\_validation\_test.sol} - Testes unitários
        \item \texttt{README.md} - Documentação do projeto
    \end{itemize}
\end{enumerate}

\subsection{Processo de Deploy}

\subsubsection{Configuração do Ambiente}

\begin{enumerate}
    \item Acesse o Remix IDE
    \item Crie uma nova pasta para o projeto
    \item Importe os arquivos do sistema obtidos do repositório GitHub
\end{enumerate}

\subsubsection{Compilação}

\begin{enumerate}
    \item Selecione a aba "Solidity Compiler"
    \item Configure a versão do compilador para 0.8.x
    \item Compile o contrato \texttt{Crid\_Validation.sol}
    \item Verifique se não há erros de compilação
\end{enumerate}

\subsubsection{Deploy}

\begin{enumerate}
    \item Conecte sua carteira MetaMask à rede desejada
    \item Selecione a aba "Deploy & Run Transactions"
    \item Configure o ambiente (Remix VM, Injected Web3, etc.)
    \item Deploy o contrato \texttt{CRID}
    \item Anote o endereço do contrato deployado
\end{enumerate}

\subsubsection{Configuração Inicial}

Após o deploy, configure o sistema:

\begin{enumerate}
    \item Execute as funções de cadastro na seguinte ordem:
    \begin{itemize}
        \item \texttt{adicionarCoordenador(address)}
        \item \texttt{adicionarOrientador(address)}
        \item \texttt{cadastrarEstudante(...)}
        \item \texttt{cadastrarDisciplina(...)}
    \end{itemize}
\end{enumerate}

\section{Demonstração Prática}

\subsection{Cenário de Teste}

Para demonstrar o funcionamento do sistema, considere o seguinte cenário:

\begin{enumerate}
    \item Um estudante \texttt{João Silva} deseja se inscrever na disciplina \texttt{EEL740}
    \item O orientador deve dar concordância ao pedido
    \item O coordenador da disciplina deve processar o pedido
\end{enumerate}

\subsection{Fluxo de Execução}

\begin{lstlisting}[caption=Exemplo de Uso Completo]
// 1. Configuração inicial (Admin)
crid.adicionarCoordenador(0x123...);
crid.adicionarOrientador(0x456...);

// 2. Cadastro de estudante
crid.cadastrarEstudante(
    0x789...,           // endereço
    "2020123456",       // matrícula
    "João Silva",       // nome
    120,                // COA
    85,                 // CRA
    6,                  // período
    0x456...            // orientador
);

// 3. Cadastro de disciplina
crid.cadastrarDisciplina(
    "EEL740",           // código
    "COMUNICACOES II",  // nome
    "EL1",              // turma
    60,                 // carga horária
    30,                 // vagas
    0x123...            // coordenador
);

// 4. Estudante faz pedido
crid.realizarPedido("EEL740");

// 5. Orientador dá concordância
crid.darConcordancia(1);

// 6. Coordenador processa pedido
crid.processarPedido(1, StatusPedido.Efetivado);
\end{lstlisting}

\subsection{Limitações Identificadas}

Durante o desenvolvimento, foram identificadas algumas limitações:

\begin{itemize}
    \item \textbf{Escalabilidade:} O sistema armazena todos os dados on-chain, o que pode ser custoso para grandes volumes
    \item \textbf{Pré-requisitos:} Não implementa validação de pré-requisitos entre disciplinas
    \item \textbf{Conflitos de Horário:} Não verifica conflitos de horário entre disciplinas
    \item \textbf{Período de Inscrição:} Não implementa controle de períodos de inscrição
    \item \textbf{Custos de Gas:} Operações podem ser custosas em mainnet
\end{itemize}

\section{Conclusão}

\subsection{Objetivos Alcançados}

O projeto CRID foi desenvolvido com sucesso, cumprindo todos os objetivos estabelecidos. O sistema demonstra a viabilidade de utilizar tecnologia blockchain para automatizar processos acadêmicos, garantindo transparência, segurança e auditabilidade.

A implementação resultou em um smart contract, com sistema de testes e documentação técnica. O uso de ferramentas de IA durante o desenvolvimento provou ser valioso para acelerar o processo e melhorar a qualidade do código.

\subsection{Aprendizados}

Durante o desenvolvimento do projeto, foram adquiridos conhecimentos importantes:

\begin{itemize}
    \item \textbf{Solidity:} Conhecimentos da linguagem e padrões de desenvolvimento
    \item \textbf{Blockchain:} Compreensão dos principais conceitos de blockchain
    \item \textbf{Smart Contracts:} Experiência prática em desenvolvimento de contratos
    \item \textbf{Testes:} Importância de testes unitários
\end{itemize}

\subsection{Trabalhos Futuros}

Possíveis melhorias e extensões para o sistema:

\begin{itemize}
    \item \textbf{Interface Web:} Desenvolvimento de interface de usuário
    \item \textbf{Otimização de Gas:} Implementação de padrões de otimização
    \item \textbf{Integração IPFS:} Armazenamento de dados não críticos off-chain
    \item \textbf{Oráculos:} Integração com sistemas externos
    \item \textbf{Governança:} Implementação de sistema de governança descentralizada
    \item \textbf{Tokens:} Sistema de incentivos baseado em tokens
\end{itemize}

\subsection{Contribuições}

Este projeto contribui para:

\begin{itemize}
    \item \textbf{Conhecimento Acadêmico:} Demonstração prática de aplicação blockchain
\end{itemize}

\subsection{Considerações Finais}

O Sistema CRID representa um passo importante na direção da digitalização e modernização dos processos acadêmicos. A utilização de blockchain não apenas resolve problemas técnicos, mas também promove valores importantes como transparência, descentralização e confiança.

O projeto demonstra que é possível aplicar tecnologias emergentes para resolver problemas reais do mundo acadêmico, criando soluções que beneficiam todos os stakeholders envolvidos: estudantes, professores, coordenadores e a instituição como um todo.

A experiência adquirida durante o desenvolvimento deste projeto será valiosa para futuros trabalhos na área de blockchain e sistemas distribuídos, contribuindo para o avanço do conhecimento e da tecnologia.

\section{Referências e Recursos}

\begin{itemize}
    \item \textbf{Repositório do Projeto:} \url{https://github.com/Alan-oliveir/cripd_validation}
    \item \textbf{Documentação Solidity:} \url{https://docs.soliditylang.org/}
    \item \textbf{Remix IDE:} \url{https://remix.ethereum.org/}
    \item \textbf{Ethereum Documentation:} \url{https://ethereum.org/developers/docs/}
\end{itemize}

\end{document}